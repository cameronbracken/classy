\documentclass[11pt,twoside]{article}

%%%%%%%%%%%%%%%%%%%%%%%%%%%
%
%  Font
%
%%%%%%%%%%%%%%%%%%%%%%%%%%%
\usepackage[sc]{mathpazo}
\linespread{1.05}         % Palatino needs more leading (space between lines)


%%%%%%%%%%%%%%%%%%%%%%%%%%%
%
%  Page Layout
%
%%%%%%%%%%%%%%%%%%%%%%%%%%%
\usepackage[margin=1in]{geometry} %changes margins
%\usepackage[parfill]{parskip} % begin paragraphs with an empty line not indent
\usepackage{multicol}


%%%%%%%%%%%%%%%%%%%%%%%%%%%
%
%  Page Header
%
%%%%%%%%%%%%%%%%%%%%%%%%%%%
\usepackage{fancyhdr}
\pagestyle{fancy}
\fancyhead{}
\fancyfoot{}
\renewcommand{\headrulewidth}{0pt}
\renewcommand{\footrulewidth}{0pt}
\fancyhead[LE,RO]{\thepage}   %page numbers

\fancyhead[CE]{\small CVEN 5313 FALL 2010}
\fancyhead[CO]{\small Problem Set 4}


%%%%%%%%%%%%%%%%%%%%%%%%%%%
%
%  Mathematics
%
%%%%%%%%%%%%%%%%%%%%%%%%%%%
\usepackage{amsmath,amssymb,amsthm}
\usepackage{cancel}
%\newcommand{\p}[2]{\frac{\partial#1}{\partial#2}}

% roman numerals
\makeatletter
\newcommand{\rmnum}[1]{\romannumeral #1}
\newcommand{\Rmnum}[1]{\expandafter\@slowromancap\romannumeral #1@}
\makeatother

\renewcommand{\d}{\partial}
%\newcommand{\vect}[1]{\underbar{#1}}
\newcommand{\vect}[1]{\vec{#1}}
\newcommand{\grad}{\nabla}
\newcommand{\tensor}[1]{\underline{\underline{#1}}}
\newcommand{\cross}{\times}
\newcommand{\curl}{\mbox{curl}}
\newcommand{\gradf}{\mbox{grad}}
\newcommand{\divf}{\mbox{div}}
\newcommand{\inline}[1]{\mbox{$#1$}}


\begin{document}


\thispagestyle{empty}
\noindent\textbf{Cameron Bracken}\\
CVEN 5313, Fall 2010\\
Problem Set 4


\begin{enumerate}

%%%%%%%%%%%%%%%%%%%%%%%%%%%%%%%%%%%%%%%%%%%%%
%
% Problem 1
%
%%%%%%%%%%%%%%%%%%%%%%%%%%%%%%%%%%%%%%%%%%%%%
\item 
\begin{enumerate}

\item $ \vect{a} \cdot \nabla \vect{a} = a_{i}\d _{j}a_{i} $
\item $ (\vect{a}\cdot\nabla)\vect{a} = a_{i}\d _{j}a_{i} $
\item $ \vect{a}\grad\vect{a} = a_{j}\d_{j}a_{i} $
\item $ \tensor{A}:\tensor{A} = A_{ij}A_{ji} $
\item $ \tensor{A}:\tensor{A}^{T} = A_{ij}A_{ij} $
\item $ \curl(\gradf(\divf(\vect{a}\cross\vect{b}))) = \epsilon_{nml}\d_{m}\d_{l}\d_{i}(\epsilon_{ijk}a_{j}b_{k}) $
\end{enumerate}

%%%%%%%%%%%%%%%%%%%%%%%%%%%%%%%%%%%%%%%%%%%%%
%
% Problem 2
%
%%%%%%%%%%%%%%%%%%%%%%%%%%%%%%%%%%%%%%%%%%%%%
\item $\tensor{\smash[b]{Q}}:\tensor{R} = \tensor{R}:\tensor{\smash[b]{Q}} = 0$ if $\tensor{\smash[b]{Q}}$ is symmetric and $\tensor{R}$ is antisymmetric. 

\textit{Proof} Wrting the tensors in index notation, $Q_{ij}R_{ji} = R_{ij}Q_{ji}$ by commuting the terms and renaming the dummy indicies so $\tensor{\smash[b]{Q}}:\tensor{R} = \tensor{R}:\tensor{\smash[b]{Q}}$.  By definition of symmetry and antisymmetry, $Q_{ij} = Q_{ji}$ and $R_{ij} = -R_{ji}$, so $Q_{ij}R_{ji} = - Q_{ji}R_{ij} = - Q_{ij}R_{ji} = 0. \qed$

%%%%%%%%%%%%%%%%%%%%%%%%%%%%%%%%%%%%%%%%%%%%%
%
% Problem 3
%
%%%%%%%%%%%%%%%%%%%%%%%%%%%%%%%%%%%%%%%%%%%%%
\item $A_{ijkl}B_{jklm}=0$ if $\tensor{A}$ is symmetric with respect to indicies $j$ and $k$ and $\tensor{B}$ is antisymemtric with respect to $j$ and $k$.

\textit{Proof} By the definitions of symetry and antisymmetry, $A_{ijkl}= A_{ikjl}$ and $B_{jklm}=-B_{kjlm}$, so $A_{ijkl}B_{jklm} = - A_{ikjl}B_{kjlm} = -A_{ijkl}B_{jklm} = 0. \qed$

%%%%%%%%%%%%%%%%%%%%%%%%%%%%%%%%%%%%%%%%%%%%%
%
% Problem 4
%
%%%%%%%%%%%%%%%%%%%%%%%%%%%%%%%%%%%%%%%%%%%%%
\item If $\nabla\rho=0 $ then $ \nabla\cdot(\rho\vect{u}\vect{u}) = \rho\vect{u}\cdot\nabla\vect{u} + \rho\vect{u}(\nabla\cdot\vect{u})$.

\textit{Proof} In index notation $\nabla\rho=\d_i\rho = 0$. In index notation, apply the product rule twice

\begin{align*}
\d_i(\rho u_iu_j) &= \rho u_i\d_i u_j +u_j\d_i(\rho u_i)\\
&= \rho u_i\d_i u_j +u_j\d_i(\rho u_i)\\
&= \rho u_i\d_i u_j +u_j\rho\d_i u_i + u_ju_i\cancelto{0}{\d_i\rho}&&\\
& = \rho\vect{u}\cdot\nabla\vect{u} + \rho\vect{u}(\nabla\cdot\vect{u})&&\qed
\end{align*}




%%%%%%%%%%%%%%%%%%%%%%%%%%%%%%%%%%%%%%%%%%%%%
%
% Problem 5
%
%%%%%%%%%%%%%%%%%%%%%%%%%%%%%%%%%%%%%%%%%%%%%
\item $a_i\d_ja_i=\grad(\frac{1}{2}\vect{a}\cdot\vect{a})$. 

\textit{Proof} 
\begin{align*}
a_i\d_ja_i &= \inline{\frac{1}{2}}\left(a_i\d_ja_i + a_i\d_ja_i\right) && \\
           &= \inline{\frac{1}{2}}\d_j(a_ia_i) &&\text{Reverse product rule} \\
           &= \d_j(\inline{\frac{1}{2}}a_ia_i) &&\\
           &= \grad\left(\inline{\frac{1}{2}}\vect{a}\cdot\vect{a}\right) &&\qed
\end{align*}

%%%%%%%%%%%%%%%%%%%%%%%%%%%%%%%%%%%%%%%%%%%%%
%
% Problem 6
%
%%%%%%%%%%%%%%%%%%%%%%%%%%%%%%%%%%%%%%%%%%%%%
\item $\curl(\gradf(\phi)) = 0$. 

\textit{Proof} 
\begin{align*}
\curl(\gradf(\phi)) & = \grad\cross(\grad\phi)\\
                    & = \epsilon_{ijk}\d_j(\d_k\phi)\\
                    & = - \epsilon_{ikj}\d_j\d_k\phi\\
                    & = - \epsilon_{ijk}\d_k\d_j\phi\\
                    & = 0 && \qed
\end{align*}

%%%%%%%%%%%%%%%%%%%%%%%%%%%%%%%%%%%%%%%%%%%%%
%
% Problem 7
%
%%%%%%%%%%%%%%%%%%%%%%%%%%%%%%%%%%%%%%%%%%%%%
\item $\vect{a}\cross(\vect{b}\cross\vect{c}) = (\vect{a}\cdot\vect{c})\vect{b} - (\vect{a}\cdot\vect{b})\vect{c}$

\textit{Proof}
\begin{align*}
\vect{a}\cross(\vect{b}\cross\vect{c}) & = \epsilon_{ijk}a_j(\epsilon_{klm}b_lc_m)\\
                                       & = \epsilon_{ijk}\epsilon_{klm}a_jb_lc_m\\
                                       & = (\delta_{il}\delta_{jm}-\delta_{im}\delta_{jl})a_jb_lc_m\\
                                       & = \delta_{il}\delta_{jm}a_jb_lc_m-\delta_{im}\delta_{jl}a_jb_lc_m 
                                         && \text{Enforce $l=i$, $j=m$ on the first term}\\
                                       & && \text{and $l=j$ and $m=i$ on the second}\\
                                       & = a_jb_ic_j-a_jb_jc_i \\
                                       & = a_ic_ib_j-a_ib_ic_j \\
                                       & = (\vect{a}\cdot\vect{c})\vect{b} - (\vect{a}\cdot\vect{b})\vect{c} && \qed
\end{align*}



%%%%%%%%%%%%%%%%%%%%%%%%%%%%%%%%%%%%%%%%%%%%%
%
% Problem 8
%
%%%%%%%%%%%%%%%%%%%%%%%%%%%%%%%%%%%%%%%%%%%%%
\item $(\vect{a}\cross\vect{b})\cdot(\vect{c}\cross\vect{d}) = (\vect{a}\cdot\vect{c})(\vect{b}\cdot{\vect{c}})-(\vect{a}\cdot\vect{d})(\vect{b}\cdot{\vect{c}})$

\textit{Proof} 
\begin{align*}
(\vect{a}\cross\vect{b})\cdot(\vect{c}\cross\vect{d}) 
            & = (\epsilon_{ijk}a_jb_k)(\epsilon_{ilm}c_ld_m)\\
            & = (\delta_{il}\delta_{jm}-\delta_{im}\delta_{jl})a_jb_kc_ld_m\\  
            & = \delta_{jl}\delta_{km}a_jb_kc_ld_m-\delta_{jm}\delta_{kl}a_jb_kc_ld_m
              && \text{Enforce $l=j$, $k=m$ on the first term}\\
            & && \text{and $l=k$ and $m=j$ on the second}\\
            & = a_jb_kc_jd_k-a_jb_kc_kd_j\\
            & = a_jc_jb_kd_k-a_jd_jb_kc_k\\
            & = (a_jc_j)(b_kd_k)-(a_jd_j)(b_kc_k)\\
            & = (\vect{a}\cdot\vect{c})(\vect{b}\cdot{\vect{c}})-(\vect{a}\cdot\vect{d})(\vect{b}\cdot{\vect{c}}) && \qed
\end{align*}


%%%%%%%%%%%%%%%%%%%%%%%%%%%%%%%%%%%%%%%%%%%%%
%
% Problem 9
%
%%%%%%%%%%%%%%%%%%%%%%%%%%%%%%%%%%%%%%%%%%%%%
\item $\grad\cdot(\grad\cross\vect{a})=0$

\textit{Proof}
\begin{align*}
\grad\cdot(\grad\cross\vect{a}) & = \d_i(\epsilon_{ijk}\d_ja_k)\\
                                & =  \epsilon_{ijk}\d_i\d_ja_k\\
                                & = -\epsilon_{jik}\d_i\d_ja_k\\
                                & = -\epsilon_{ijk}\d_j\d_ia_k && \text{Rename Indicies}\\
                                & = -\epsilon_{ijk}\d_i\d_ja_k && \text{Reverse order of derivatives}\\
                                & = -\d_i(\epsilon_{ijk}\d_ja_k)\\
                                & = 0 && \qed\\
\end{align*}


%%%%%%%%%%%%%%%%%%%%%%%%%%%%%%%%%%%%%%%%%%%%%
%
% Problem 10
%
%%%%%%%%%%%%%%%%%%%%%%%%%%%%%%%%%%%%%%%%%%%%%
\item $\grad\cdot(\vect{a}\cross\vect{a})=\vect{b}\cdot(\grad\cross\vect{a})-\vect{a}\cdot(\grad\cross\vect{b})$

\textit{Proof} 
\begin{align*}
\grad\cdot(\grad\cross\vect{a}) & = \d_i(\epsilon_{ijk}a_jb_k)\\
                                & = \epsilon_{ijk}\d_i\d_ja_k\\
                                & = \epsilon_{ijk}(a_j\d_ib_k+b_k\d_ia_j) && \text{Product rule}\\
                                & = a_j\epsilon_{ijk}\d_ib_k+b_k\epsilon_{ijk}\d_ia_j\\
                                & = -a_j\epsilon_{jik}\d_ib_k+b_k\epsilon_{kij}\d_ia_j\\
                                & = b_k\epsilon_{kij}\d_ia_j-a_j\epsilon_{jik}\d_ib_k\\
                                & = b_i\epsilon_{ijk}\d_ja_k-a_i\epsilon_{ijk}\d_jb_k &&\text{Renaming indicies}\\
                                & = \vect{b}\cdot(\grad\cross\vect{a})-\vect{a}\cdot(\grad\cross\vect{b})&& \qed
\end{align*}


%%%%%%%%%%%%%%%%%%%%%%%%%%%%%%%%%%%%%%%%%%%%%
%
% Problem 11
%
%%%%%%%%%%%%%%%%%%%%%%%%%%%%%%%%%%%%%%%%%%%%%
\item $\grad\cross(\vect{a}\cross\vect{b}) = \vect{a}(\grad\cdot\vect{b})+\vect{b}\cdot\grad\vect{a}-\vect{a}\grad\vect{b}-\vect{b}(\grad\cdot\vect{a})$

\textit{Proof}
\begin{align*}
\grad\cross(\vect{a}\cross\vect{b}) & = \epsilon_{ijk}\d_j(\epsilon_{klm}a_lb_m)\\
                                    & = \epsilon_{ijk}\epsilon_{klm}\d_j(a_lb_m)\\
                                    & = (\delta_{il}\delta_{jm}-\delta_{im}\delta_{jl})(a_l\d_jb_m+b_m\d_ja_l)\\
                                    & = \delta_{il}\delta_{jm}a_l\d_jb_m-\delta_{im}\delta_{jl}a_l\d_jb_m+ \delta_{il}\delta_{jm}b_m\d_ja_l - \delta_{im}\delta_{jl}b_m\d_ja_l\\
                                    & = a_i\d_jb_j-a_j\d_jb_i+b_j\d_ja_i-b_i\d_ja_j &&\text{Enforce right to left}\\
                                    & = \vect{a}(\grad\cdot\vect{b})+\vect{b}\cdot\grad\vect{a}-\vect{a}\grad\vect{b}-\vect{b}(\grad\cdot\vect{a}) && \qed
\end{align*}


%%%%%%%%%%%%%%%%%%%%%%%%%%%%%%%%%%%%%%%%%%%%%
%
% Problem 12
%
%%%%%%%%%%%%%%%%%%%%%%%%%%%%%%%%%%%%%%%%%%%%%
\item $\grad[(\vect{u}\cross\vect{v})\cross\vect{w}] = \vect{w}\cdot[(\vect{v}\cdot\grad)\vect{u}-(\vect{u}\cdot\grad)\vect{v}]$ \\
given \\ 
$\divf(\vect{u})=\grad\cdot\vect{u}=\d_iu_i=0$\\ 
$\divf(\vect{v})=\grad\cdot\vect{v}=\d_iv_i=0$\\ 
$\curl(\vect{w})=\grad\cross\vect{w}=\epsilon_{ijk}\d_jw_k=0$

\textit{Proof}
\begin{align*}
\grad[(\vect{u}\cross\vect{v})\cross\vect{w}] 
            & = && \d_l[(\epsilon_{ijk}u_jv_k)\epsilon_{lim}w_m]\\
            & = && \epsilon_{ijk}\epsilon_{lim}\d_l(u_jv_kw_m)\\
            & = && \epsilon_{ijk}\epsilon_{lim}[u_j\d_l(v_kw_m)+v_kw_m\d_lu_j]\\
            & = && \epsilon_{ijk}\epsilon_{lim}(u_jv_k\cancelto{0}{\d_lw_m}+u_jw_m\d_lv_k+v_kw_m\d_lu_j)\\
            & = &&  (\delta_{jm}\delta_{kl}-\delta_{jl}\delta_{km})(u_jw_m\d_lv_k+v_kw_m\d_lu_j)\\
            & = && \delta_{jm}\delta_{kl}u_jw_m\d_lv_k-\delta_{jl}\delta_{km}u_jw_m\d_lv_k\\
            &   &&+ \delta_{jm}\delta_{kl}v_kw_m\d_lu_j-\delta_{jl}\delta_{km}v_kw_m\d_lu_j 
                && \text{Enforce right to left}\\
            & = && u_jw_j\cancelto{0}{\d_kv_k}-u_jw_k\d_jv_k+v_kw_j\d_ku_j-v_kw_k\cancelto{0}{\d_ju_j}\\
            & = && w_jv_k\d_ku_j-u_kw_j\d_kv_j && \text{Rename dummy indicies}\\
            & = && w_j[v_k\d_ku_j-u_k\d_kv_j]\\
            & = && \vect{w}\cdot[(\vect{v}\cdot\grad)\vect{u}-(\vect{u}\cdot\grad)\vect{v}] && \qed
\end{align*}


%%%%%%%%%%%%%%%%%%%%%%%%%%%%%%%%%%%%%%%%%%%%%
%
% Problem 13
%
%%%%%%%%%%%%%%%%%%%%%%%%%%%%%%%%%%%%%%%%%%%%%
\item $\grad\cross[\grad(\inline{\frac{1}{2}}\vect{a}\cdot\vect{a})-\vect{a}\cross(\grad\cross\grad\vect{a})] = \grad\cross(\vect{a}\cdot\grad\vect{a})$

\textit{Proof}
\begin{align*}
\grad\cross[\grad(\inline{\frac{1}{2}}\vect{a}\cdot\vect{a})-\vect{a}\cross(\grad\cross\grad\vect{a})] 
   & = \epsilon_{npi}\d_p[\d_i(\inline{\frac{1}{2}}a_ja_j)-\epsilon_{ijk}a_j(\epsilon_{klm}\d_la_m)]\\
   & = \epsilon_{npi}\d_p[a_j\d_ia_j-\epsilon_{ijk}\epsilon_{lmk}a_j\d_la_m] && \text{By problem 5}\\
   & = \epsilon_{npi}\d_p[a_j\d_ia_j- (\delta_{il}\delta_{jm}-\delta_{im}\delta_{jl})a_j\d_la_m]\\
   & = \epsilon_{npi}\d_p[a_j\d_ia_j - \delta_{il}\delta_{jm}a_j\d_la_m+\delta_{im}\delta_{jl}a_j\d_la_m] && \text{Enforce right to left}\\
   & = \epsilon_{npi}\d_p[a_j\d_ia_j - a_j\d_ia_j + a_j\d_ja_i] \\
   & = \epsilon_{npi}\d_p(a_j\d_ja_i) \\
   & = \grad\cross(\vect{a}\cdot\grad\vect{a}) && \qed\\
\end{align*}



%%%%%%%%%%%%%%%%%%%%%%%%%%%%%%%%%%%%%%%%%%%%%
%
% Problem 14
%
%%%%%%%%%%%%%%%%%%%%%%%%%%%%%%%%%%%%%%%%%%%%%
\item $\grad\cross(\vect{a}\cdot\grad\vect{a}) = \vect{a}\cdot(\grad\cross\vect{a})+(\grad\cdot\vect{a})(\grad\cross\vect{a})- (\grad\cross\vect{a})\cdot(\grad\vect{a})$\\
given \\
$\grad\cross[\grad(\inline{\frac{1}{2}}\vect{a}\cdot\vect{a})-\vect{a}\cross(\grad\cross\grad\vect{a})] = \grad\cross(\vect{a}\cdot\grad\vect{a})$

\textit{Proof}
\begin{align*}
\grad\cross(\vect{a}\cdot\grad\vect{a}) 
       & = \grad\cross[\grad(\inline{\frac{1}{2}}\vect{a}\cdot\vect{a})-\vect{a}\cross(\grad\cross\grad\vect{a})] \\
       & = \cancelto{0}{\grad\cross[\grad(\inline{\frac{1}{2}}\vect{a}\cdot\vect{a})]}-\grad\cross[\vect{a}\cross(\grad\cross\grad\vect{a})] && \text{By problem 6}\\
       & = - \grad\cross[\vect{a}\cross(\grad\cross\grad\vect{a})]\\
       & = \epsilon_{pnl}\d_n(\epsilon_{lmi}a_m \epsilon_{ijk}\d_ja_k) \\
\end{align*}


\end{enumerate}






\end{document}
