\documentclass[11pt,twoside]{article}

%%%%%%%%%%%%%%%%%%%%%%%%%%%
%
%  Font
%
%%%%%%%%%%%%%%%%%%%%%%%%%%%
\usepackage[sc]{mathpazo}
\linespread{1.05}         % Palatino needs more leading (space between lines)


%%%%%%%%%%%%%%%%%%%%%%%%%%%
%
%  Page Layout
%
%%%%%%%%%%%%%%%%%%%%%%%%%%%
\usepackage[margin=1in]{geometry} %changes margins
%\usepackage[parfill]{parskip} % begin paragraphs with an empty line not indent
\usepackage{multicol}


%%%%%%%%%%%%%%%%%%%%%%%%%%%
%
%  Page Header
%
%%%%%%%%%%%%%%%%%%%%%%%%%%%
\usepackage{fancyhdr}
\pagestyle{fancy}
\fancyhead{}
\fancyfoot{}
\renewcommand{\headrulewidth}{0pt}
\renewcommand{\footrulewidth}{0pt}
\fancyhead[LE,RO]{\thepage}   %page numbers

\fancyhead[CE]{\small CVEN 5313 FALL 2010}
\fancyhead[CO]{\small Problem Set 5}


%%%%%%%%%%%%%%%%%%%%%%%%%%%
%
%  Mathematics
%
%%%%%%%%%%%%%%%%%%%%%%%%%%%
\usepackage{amsmath,amssymb,amsthm}
\usepackage{cancel}
\newcommand{\p}[2]{\frac{\partial#1}{\partial#2}}

% roman numerals
\makeatletter
\newcommand{\rmnum}[1]{\romannumeral #1}
\newcommand{\Rmnum}[1]{\expandafter\@slowromancap\romannumeral #1@}
\makeatother

\renewcommand{\d}{\partial}
%\newcommand{\vect}[1]{\underbar{#1}}
\newcommand{\vect}[1]{\vec{#1}}
\newcommand{\grad}{\nabla}
\newcommand{\tensor}[1]{\underline{\underline{#1}}}
\newcommand{\cross}{\times}
\newcommand{\curl}{\mbox{curl}}
\newcommand{\gradf}{\mbox{grad}}
\newcommand{\divf}{\mbox{div}}
\newcommand{\inline}[1]{\mbox{$#1$}}

\usepackage[pdftex,bookmarks,colorlinks,breaklinks]{hyperref}
\hypersetup{linkcolor=black,citecolor=black,filecolor=black,urlcolor=black}
\begin{document}


\thispagestyle{empty}
\noindent\textbf{Cameron Bracken}\\
CVEN 5313, Fall 2010\\
Problem Set 5


\begin{enumerate}

%%%%%%%%%%%%%%%%%%%%%%%%%%%%%%%%%%%%%%%%%%%%%
%
% Problem 1
%
%%%%%%%%%%%%%%%%%%%%%%%%%%%%%%%%%%%%%%%%%%%%%
\item 
\begin{enumerate}

%%%%%%%%%%%%%%%%%%%%%%%%%%%%%%%%%%%%%%%%%%%%%
%%%%%%%%%%%%%%%%%%%%%%%%%%%%%%%%%%%%%%%%%%%%%
\item In time $dt$ a differential element $dS$ expands or contracts at a rate $\vect{u}$ in the direction $\hat{n}$ a distance a distance $d\ell$. That is 

$$d\ell=\vect{u}\cdot\hat{n}\,dS\,dt.$$

Integrating (summing up) all the differential elements on the surface 

$$\iint_{S(t)}d\ell=\iint_{S(t)}\vect{u}\cdot\hat{n}\,dS\,dt$$
$$\frac{dV(t)}{dt}=\iint_{S(t)}\vect{u}\cdot\hat{n}\,dS$$

%%%%%%%%%%%%%%%%%%%%%%%%%%%%%%%%%%%%%%%%%%%%%
%%%%%%%%%%%%%%%%%%%%%%%%%%%%%%%%%%%%%%%%%%%%%
\item 
$$\frac{dV(t)}{dt} = \iiint_{\mathcal{R}(t)}\grad\cdot\vect{u}\,dV$$
From part (a)
$$\frac{dV(t)}{dt} = \iint_{S(t)}\vect{u}\cdot\hat{n}\,dS = \iint_{S(t)}\vect{u}\cdot\,d\vect{S}$$
Where $\hat{n}\,dS=d\vect{S}$. So by analogy to the divergence theorem ($\vect{u}=\tensor{T}$)
$$\frac{dV(t)}{dt} = \iiint_{\mathcal{R}(t)}\grad\cdot\vect{u}\,dV$$

\item $\grad\cdot\vect{u}$ can be interpreted as the expansion or contraction of a fluid volume due to the velocity field. 
\end{enumerate}

%%%%%%%%%%%%%%%%%%%%%%%%%%%%%%%%%%%%%%%%%%%%%
%
% Problem 2
%
%%%%%%%%%%%%%%%%%%%%%%%%%%%%%%%%%%%%%%%%%%%%%

\item 
Consider a function $f(x_1,t)$ defined over a region $\mathcal{R}$.  The three dimensional Liebnitz formula is: 
\begin{equation}
\frac{d}{dt}\iiint_{R(t)}f(x_1,t)dV = \iiint_V \p{f}{t} dV + \iint_A n_k w_k f\: dA\label{lieb}
\end{equation}
Exmaining the first term on the left hand side, 
\begin{align}
\frac{d}{dt}\iiint_{R(t)}f(x_1,t)dV &= \frac{d}{dt}\iiint_{R(t)}f(x_1,t)dx_1dx_2dx_3\nonumber\\
                                    &= \frac{d}{dt}\int_{a(t)}^{b(t)}f(x_1,t)dx_1\iint_Adx_2dx_3\nonumber\\
                                    &= A\frac{d}{dt}\int_{a(t)}^{b(t)}f(x_1,t)dx_1\label{lhs}
\end{align}

Now examine the first term on the right hand side of \autoref{lieb}
\begin{align}
\iiint_V \p{f}{t} dV &= \iiint_V \p{f}{t} dx_1dx_2dx_3\nonumber\\
&=A\int_a^b \p{f}{t} dx_1\label{rhs1}
\end{align}

Next, examine the second term on the right hand side of \autoref{lieb}. Use a rectangular region with face areas $A_A=A_B=A$ (in the $x_2-x_3$ plane). Since the other faces have normal vectors orthoginal to the $x_1$ direction we dont need to consider them because $f$ is only a function of $x_1$ so 

\begin{align}
\iint_A n_k w_k f\: dA &= \iint_{A_A} n_k w_k f\: dx_2\,dx_3 + \iint_{A_B} n_k w_k f\: dx_2\,dx_3 \nonumber\\
                       &= \iint_{A_A} (-1) \frac{da}{dt} f (x_1,t)\: dx_2\,dx_3 + \iint_{A_B} (1) \frac{db}{dt} f (x_1,t)\: dx_2\,dx_3 \nonumber\\
                       &=  - \frac{da}{dt} f (a,t)\iint_{A_A}\: dx_2\,dx_3 + \frac{db}{dt} f (b,t)\iint_{A_B}\: dx_2\,dx_3 \nonumber\\
                       & = \frac{db}{dt} f (b,t) A - \frac{da}{dt} f (a,t)A\label{rhs2}
\end{align}

Finally combine Equations \ref{lhs}, \ref{rhs1} and \ref{rhs2}

$$A\frac{d}{dt}\int_{a(t)}^{b(t)}f(x_1,t)dx_1 = A\int_a^b \p{f}{t} dx_1 + \frac{db}{dt} f (b,t) A - \frac{da}{dt} f (a,t)A$$

And divide by $A$

$$\frac{d}{dt}\int_{a(t)}^{b(t)}f(x_1,t)dx_1 = \int_a^b \p{f}{t} dx_1 + \frac{db}{dt} f (b,t)  - \frac{da}{dt} f (a,t).$$


%%%%%%%%%%%%%%%%%%%%%%%%%%%%%%%%%%%%%%%%%%%%%
%
% Problem 3
%
%%%%%%%%%%%%%%%%%%%%%%%%%%%%%%%%%%%%%%%%%%%%%

\end{enumerate}






\end{document}
