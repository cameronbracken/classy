\documentclass[11pt]{article}
%\usepackage[left=2cm,right=2cm,top=3cm,bottom=2cm]{geometry}
%\usepackage[parfill]{parskip}
%\usepackage{graphicx}
%\usepackage[usenames]{colors}
\usepackage{mathpazo}
\usepackage{geometry}
\usepackage{natbib}
\bibliographystyle{agufull04}
\usepackage[parfill]{parskip}


\begin{document}

{\large\bf Second-year forecasts using the multiscale variability of paleo reconstructions}\\
Cameron Bracken\\
CVEN 5333

Nearly 80\% of flow in the Colorado River Basin is due to snowmelt which is a strongly seasonal phenomenon.  70\% of the flow (by volume) on average occurs in the months of April-July.  Current operational forecasting methods used by the Colorado River Forecast Center (CBRFC) use antecedent flow and soil moisture conditions, snowpack and climate indicies \citep{Brandon2005}.  Experimental methods include large scale climate drivers and obtain similar results \citep{Bracken:2010p2682, Regonda2006, Grantz2005}.  These methods, especially those using snowpack information, are very skilful in predicting seasonal flow volumes.  

Despite the skill of these methods, they only provide reasonable results in the first year.  Longer term informaton, especially for large reservoirs, is useful to water managers in terms of planning. Every month the Bureau of Reclamation (BOR) updates their reservoir forecast model, the ``24 Month Study.''  This model runs for approximately 24 months and produces water availaibility information for all the major reservoirs in the colorado river basin.  The 24 Month Study relies on CBRFC inflow forecasts for the first year of operations but the second year (the out year) it uses average historial values (climatology).  As a result the reservoir outlook for the second year always tends toward the optimistic. 

Since traditional physical models cannot provide skilful forecasts in the out year, other methods mus be explored.  The approxamately 100 year record of flow at Lees Ferry is relatively short from the perspective of Multiscale Hydrology.  The paleo record of streamflow privides an attractive source of this lonterm information.  Multiple reconstructions are available at Lees Ferry that span 500-1000 years \citep{Woodhouse:2006p1287, Gangopadhyay:2009p61}.  The paleo record provides a rich set of extreme values not seen in the historic record. 

I propose to exploit the rich variability in the paleo record of flows at Lees Ferry to obtain out-year forecasts.  I hypothesize that the large size of the Colorado Basin and its proven dependence on large scale climate indicies will lead to persistance in the state of the system. That is if the system previously was above there average for the last three years, than it may possibly tend to be  above average for the next two years or visa versa.  The methodology involves developing all such possible transitions from the paleo data and subsequently using this information to simulate the trasition to future states in the historic record.  Future states are simulated based on a trasitison probability (markov chain) and a nearness in magnitude to the previous seasonal volumes (K-nearest neighbor).  

\bibliography{../references}

\end{document}  