\documentclass[11pt]{article}
\usepackage{geometry}
\usepackage[colorlinks]{hyperref}
\usepackage{graphicx}
\usepackage{mathpazo}
\usepackage{epstopdf}
\usepackage[parfill]{parskip}
\usepackage{fancyvrb}

\makeatletter
\let \@sverbatim \@verbatim
\def \@verbatim {\@sverbatim \verbatimplus}
{\catcode`'=13 \gdef \verbatimplus{\catcode`'=13 \chardef '=13 }} 
\makeatother

\newcommand{\R}{\textsf{R}}

\begin{document}

\textbf{CVEN 4333, Spring 2010, Assignment \#11, Due Friday April 16 at 5:00 in Cameron Bracken's mailbox. No late papers accepted.}

This assignemnt will be focused around:

\begin{quote}Moore, R. D. and Scott, D.F.. Camp Creek Revisited: Streamflow Changes Following Salvage Harvesting in a Medium-Sized, Snowmelt-Dominated Catchment. Canadian water Resources Journal (2005) vol. 30 (4) pp. 331-334\end{quote}

This article is a bit hard to find so download it here: \url{http://animas.colorado.edu/~bracken/CVEN4333/2005Moore.pdf}.  This article investigates the changes in the Camp Creek annual hydrograph due mainly to logging. 

\begin{enumerate}

\item Read this article and answer the following questions: 

\begin{enumerate}
\item What is the paired catchment technique? Why is this technique useful?
\item What factors in Greata Creek might confound the results?
\item Why might a small snowmelt driven basin be sensitive to land use changes?
\item What factors do the authors discuss in Camp Creek (other than logging) that may contribute to changes in the streamflow regime?
\item The authors construct paired effect models for each month as well as for annual data and for peak flow data but only the April model is statistically significant.  What reasons do the authors give for this?
\end{enumerate}

\item The data used in this study can be found on the Environment Canada website: \url{http://www.wsc.ec.gc.ca/hydat/H2O/index_e.cfm?cname=graph.cfm}.  I was really nice and downloaded this data for you.  Also the data is in a really wacky format, so I am providing a function to read it in and format it. The data for both Camp Creek and Greata Creek are daily averages in cms. You will need the files \verb"flowDaily_Camp.csv", \verb"flowDaily_Greata.csv" and \verb"read.ec.flow.R" from \url{http://animas.colorado.edu/~bracken/CVEN4333/}.

Then you can read in the data with:

\begin{verbatim}
source('read.ec.flow.R')
camp <- read.ec.flow('flowDaily_Camp.csv')
greata <- read.ec.flow('flowDaily_Greata.csv')
\end{verbatim}

This will give you two data frames with columns `y', `m', `d', and `flow' corresponding to the the year, month day and streamflow value. 

\item Make time vectors and plot both timeseries just like we did in HW 3. 

\item Extract monthly data for April so that we can reproduce the one statistically significant model in the paper:

\begin{verbatim}
    #List of years for each site
y.c <- unique(camp$y)
y.g <- unique(greata$y)
    #preallocate vectors
apr.c <- numeric(length(y.c))
apr.g <- numeric(length(y.g))

    #mean april flow at camp creek and greata
for(y in 1:length(y.c))
    apr.c[y] <- mean(camp$flow[camp$y == y.c[y] & camp$m == 4])
for(y in 1:length(y.g))
    apr.g[y] <- mean(greata$flow[greata$y == y.g[y] & greata$m == 4])

apr.c <- apr.c[y.c %in% y.g]
apr.c.pre <- apr.c[y.g<1978]
apr.g.pre <- apr.g[y.g<1978]
\end{verbatim}

\item Reproduce the results for the April model:

\begin{verbatim}
summary(lm(apr.c.pre~apr.g.pre))
\end{verbatim}

We can't reproduce the same $p$ value because we are not fitting the full model and using the same test but we can use this as a sanity check. How do your results compare?

\clearpage
\item Calcualte and plot the average annual hydrograph of mean daily flow:

\begin{verbatim}
    #Days per month
dpm <- c(31,29,31,30,31,30,31,31,30,31,30,31)
    #preallocate
am.c.pre <- am.g.pre <- am.c.post <- am.g.post <- numeric(366)
day <- 0
    #for each month, go through each day and calculate the 
    # mean for the pre and post logging periods at each site. 
for(m in 1:12){
    for(d in 1:dpm[m]){
        day <- day + 1
        this.day.c <- camp$m == m & camp$d == d
        this.day.g <- greata$m == m & greata$d == d
        am.c.pre[day] <- mean(camp$flow[this.day.c & camp$y < 1978])
        am.c.post[day] <- mean(camp$flow[this.day.c & camp$y >= 1978])
        am.g.pre[day] <- mean(greata$flow[this.day.g & greata$y < 1978])
        am.g.post[day] <- mean(greata$flow[this.day.g & greata$y >= 1978])
     }
}

\end{verbatim}

Plot both sites on the same graph, and to highlight the differences in the hydrographs during each period, fill them in:

\begin{verbatim}
plot(am.c.pre,type='l')
lines(am.g.pre,type='l')
polygon(c(1:366,366:1),c(am.c.pre,rev(am.g.pre)),col='blue')
\end{verbatim}

Do this for the post logging period as well. Add line colors and/or styles and a legend to distingush the data. 

\item Book Probelem 9.1
\item Book Probelem 9.4
\item Book Probelem 9.8 (for Boulder creek) *
\item Book Probelem 9.11
\item Book Probelem 9.26 (for Boulder creek and as compared to observational hydrograph) *
\item Book Probelem 9.37

\end{enumerate}
* For 9.8 and 9.26 use data from Friday April 9 at 12 pm. 

\end{document}  







