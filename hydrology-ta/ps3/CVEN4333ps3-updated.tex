\documentclass[11pt]{article}
\usepackage{geometry}
\usepackage[colorlinks]{hyperref}
\usepackage{graphicx}
\usepackage{mathpazo}
\usepackage[parfill]{parskip}

\begin{document}

CVEN 4333, Spring 2010, Assignment \#3, Due Thursday January 28 at 5:00 in Cameron Bracken's mailbox. No late papers accepted. 

\begin{enumerate}
\item A time series is a set of data points measured at a fixed interval in time.  In this problem you will analyze a hydrologic time series.  Please put all of your results from this problem in a document together.  If you use Word, use Insert $>$ Picture $>$ ``From File...''  when inserting PDF files, don't drag and drop.  Take a look at \url{http://animas.colorado.edu/~bracken/CVEN4333/R-intro.pdf}, we will be going through this document in class. 

\begin{enumerate}

\item Download the text file: \\\url{http://animas.colorado.edu/~bracken/CVEN4333/plant-inflow.txt},  This file contains daily average inflow time series at the wastwater treatment plant in Arcata, California.  The units are million gallons per day (MGD). Create a directory for your work on this assignment and save the data file there. 

\item Change the working directory in \textsf{R} to the directory you created containing the data file. 

\item Read this data into a data frame in \textsf{R}, use the following command:
\begin{verbatim}
dat <- read.table("plant-inflow.txt",header=T)
\end{verbatim} 
What does the \texttt{header=T} argument do (hint: use the built in help, \texttt{?read.table})?

\item Missing values in the file are indicated with a value of \texttt{-9999}. Replace these values in the data frame with \texttt{NA} (so they do not mess up the plot).  Use the command:
\begin{verbatim}
dat[dat==-9999] <- NA
\end{verbatim}

\item To plot the the inflow time series you need to combine the year month and day values like so:

\begin{verbatim}
years <- dat$year + dat$month/12 + dat$day/365
\end{verbatim}

Now plot the inflow time series (inflow versus time) and save the plot as a pdf file in the directory you created for this assignment. Use the command below to generate the plot:

\begin{verbatim}
plot(years, dat$inflow, type="l")
\end{verbatim}
You will need to modify it a bit to add labels using the \texttt{xlab}, \texttt{ylab} and \texttt{main} arguments. Qualitatively describe this data, do you see any problems from a data perspective (strange values, outliers, etc.)?

\item The City is interested in monitoring the inflow on December 4th of every year. Extract the data for this date into a vector.  
\begin{verbatim}
dec4 <- dat$inflow[dat$day == 4 & dat$month == 12]
\end{verbatim}

Calculate the mean, standard deviation, max, min, median and quartiles. The commands you need are \texttt{mean}, \texttt{sd}, \texttt{max}, \texttt{min} and \texttt{quantile}.  Display these values in a table and show the commands you used. 

\item Plot a histogram of the December 4th values with a rug plot: 

\begin{verbatim}
hist(dec4)
rug(dec4)
\end{verbatim}
Again, modify the hist command to add an x axis label and a main plot title. Save this plot as a pdf file.  Describe this histogram, do you notice anything funny?  What are two possible causes (thing about resolution and quantity of data)?

\end{enumerate} 
From the book:
\item 4.3 
\item 4.5
\item 4.14

\end{enumerate}

\end{document}  