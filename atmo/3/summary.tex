\documentclass[11pt,oneside]{article}
\usepackage{geometry}			% See geometry.pdf to learn the layout options.								
\geometry{letterpaper}			% ... or a4paper or a5paper or ... 
\usepackage[parfill]{parskip}	% Activate to begin paragraphs with an empty  line rather than an indent
\usepackage{mathpazo}
\usepackage{amsmath}		
\usepackage{graphicx}
\usepackage{hyperref}

\usepackage[font={bf,footnotesize},textfont=md,margin=30pt,aboveskip=0pt,belowskip=0pt]{caption}

\begin{document}


\begin{center}
	\textbf{Reading Summary - GEOG 5100}\\
	{\itshape Cameron Bracken} - September 27, 2009
\end{center}

The first article about the Runge-Kutta method describes several implementations and their advantages and disadvantages. The Euler method is the most efficient but least accurate of the methods and is rarely used in practice.   Similarly the second order Runge-Kutta is more accurate and less efficient and is also rarely used in practice.  Higher order Runge-Kutta methods use in-between steps to help make evaluation of the derivate at the  end of the interval more accurate. The fourth order Runge-Kutta method is often used in practice because of its reasonable balance between efficiency and accuracy. 

The second article discusses the importance of cloud feedback in atmospheric models of the early earth.  The authors suggest cloud feedback may have had a large effect on the early Earth, and neglecting them misses an important climate component.  This is an alternative to the snowball earth hypothesis.  The model used in this study is a one-dimensional radio convective model. The authors note that their parametrization of clouds may not be correct but the results indicate that reconstructing early earth climate is very uncertain. 

\end{document}
