\documentclass[11pt]{amsart}
\usepackage[margin=1.2in,top=.6in]{geometry}                % See geometry.pdf to learn the layout options. There are lots.
\geometry{letterpaper}                   % ... or a4paper or a5paper or ... 
\usepackage[parfill]{parskip}    % Activate to begin paragraphs with an empty line rather than an indent
\usepackage{setspace}

\title{Reading Summary - Week 2}
\author{Cameron Bracken}
%\date{}                                           % Activate to display a given date or no date

\begin{document}
\maketitle
\doublespacing

The first article was related specifically to the development of the Daisyworld model.  The Daisyworld model is an energy balance model with some additional complexity.  Each unit of land in the Daisyworld model is covered in either white, black or no daisies.   The daisies grow and die depending on the local temperature.  In turn the daisies affect the local temperature via their albedo.  White daisies have a high albedo and black have a low albedo (the bare ground has a different albedo as well).  Each ground type absorbs short wave radiation and emits long wave. Black daisies absorb the most short wave radiation and consequently distribute energy to colder regions.

The second article was concerned with the investigation of the Gaia hypothesis.  The theory being that the Earth and the life on the Earth (the biota) are part of a ``regulatory feedback loop'' which actively sustains conditions favorable for life.  The Daisyworld model is a simple hypothetical example of a self-regulating system.  This is opposed to the possibility that the Earth is self regulating despite the life that exists on it, and the life simply benefits from that self regulation.  The author argues at length that the Gaian self-regulation is not only present but inevitable.  For example He argues that an organism placed on a lifeless planet which alters it environment will receive either negative or positive feedback (from the environment), influencing the development of the environment-altering trait.  He calls this a ``tendency toward self regulation.''  The author goes on to add complexity to the Daisyworld model by allowing for a mutation, a gray daisy.  With this mutation, the author claims that both `Gaian' and `anti-Gaian' behavior are equally likely. Under different solar luminosities and albedos, various conditions evolve. 


\end{document}  