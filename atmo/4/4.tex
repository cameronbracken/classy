\documentclass[11pt,oneside]{article}
\usepackage{geometry}			% See geometry.pdf to learn the layout options.								
\geometry{letterpaper}			% ... or a4paper or a5paper or ... 
\usepackage[parfill]{parskip}	% Activate to begin paragraphs with an empty  line rather than an indent
\usepackage{mathpazo}
\usepackage{amsmath}		
\usepackage{graphicx}
\usepackage{hyperref}

\usepackage[font={bf,footnotesize},textfont=md,margin=30pt,aboveskip=0pt,belowskip=0pt]{caption}

\begin{document}


\begin{center}
	\textbf{Assignment 4: Lorenz equations - GEOG 5100}\\
	{\itshape Cameron Bracken} - October 5, 2009
\end{center}


\begin{enumerate}

\item Land surface parameterizations accounts for momentum, and energy fluxes. 

\item A climate model must conserve mass, energy and momentum. 

\item In a spectral model values over the surface of the earth are represented as an infinite series of wave functions. Spectral truncation occurs when any number of waves less than infinity are used to represent a value.  A T21 model uses 21 waves to represent each variable in each latitude zone at each vertical level.   The `T' stands for a particular type of truncation (triangular).


\end{enumerate}
\end{document}
